% Options for packages loaded elsewhere
\PassOptionsToPackage{unicode}{hyperref}
\PassOptionsToPackage{hyphens}{url}
%
\documentclass[
  11,
]{article}
\usepackage{amsmath,amssymb}
\usepackage{iftex}
\ifPDFTeX
  \usepackage[T1]{fontenc}
  \usepackage[utf8]{inputenc}
  \usepackage{textcomp} % provide euro and other symbols
\else % if luatex or xetex
  \usepackage{unicode-math} % this also loads fontspec
  \defaultfontfeatures{Scale=MatchLowercase}
  \defaultfontfeatures[\rmfamily]{Ligatures=TeX,Scale=1}
\fi
\usepackage{lmodern}
\ifPDFTeX\else
  % xetex/luatex font selection
\fi
% Use upquote if available, for straight quotes in verbatim environments
\IfFileExists{upquote.sty}{\usepackage{upquote}}{}
\IfFileExists{microtype.sty}{% use microtype if available
  \usepackage[]{microtype}
  \UseMicrotypeSet[protrusion]{basicmath} % disable protrusion for tt fonts
}{}
\makeatletter
\@ifundefined{KOMAClassName}{% if non-KOMA class
  \IfFileExists{parskip.sty}{%
    \usepackage{parskip}
  }{% else
    \setlength{\parindent}{0pt}
    \setlength{\parskip}{6pt plus 2pt minus 1pt}}
}{% if KOMA class
  \KOMAoptions{parskip=half}}
\makeatother
\usepackage{xcolor}
\usepackage[margin=1in]{geometry}
\usepackage{graphicx}
\makeatletter
\def\maxwidth{\ifdim\Gin@nat@width>\linewidth\linewidth\else\Gin@nat@width\fi}
\def\maxheight{\ifdim\Gin@nat@height>\textheight\textheight\else\Gin@nat@height\fi}
\makeatother
% Scale images if necessary, so that they will not overflow the page
% margins by default, and it is still possible to overwrite the defaults
% using explicit options in \includegraphics[width, height, ...]{}
\setkeys{Gin}{width=\maxwidth,height=\maxheight,keepaspectratio}
% Set default figure placement to htbp
\makeatletter
\def\fps@figure{htbp}
\makeatother
\setlength{\emergencystretch}{3em} % prevent overfull lines
\providecommand{\tightlist}{%
  \setlength{\itemsep}{0pt}\setlength{\parskip}{0pt}}
\setcounter{secnumdepth}{-\maxdimen} % remove section numbering
\usepackage{mathtools}
\usepackage[brazil]{babel}
\usepackage{graphicx}
\usepackage[a4paper]{geometry}
\usepackege{extpfeil}
\ifLuaTeX
  \usepackage{selnolig}  % disable illegal ligatures
\fi
\IfFileExists{bookmark.sty}{\usepackage{bookmark}}{\usepackage{hyperref}}
\IfFileExists{xurl.sty}{\usepackage{xurl}}{} % add URL line breaks if available
\urlstyle{same}
\hypersetup{
  pdftitle={Guía de Matemáticas III, Otoño 2023. Segundo Parcial},
  hidelinks,
  pdfcreator={LaTeX via pandoc}}

\title{Guía de Matemáticas III, Otoño 2023. Segundo Parcial}
\author{}
\date{\vspace{-2.5em}}

\begin{document}
\maketitle

\hypertarget{subespacios-vectoriales}{%
\subsection{Subespacios Vectoriales}\label{subespacios-vectoriales}}

\begin{enumerate}
\def\labelenumi{\arabic{enumi}.}
\tightlist
\item
  Determine si los siguientes conjuntos son sub-espacios vectoriales:
\end{enumerate}

\begin{enumerate}
\def\labelenumi{\alph{enumi})}
\item
  \(V = M_{22}; H = \{ A = \begin{pmatrix} 0 & a \\ -a & 0\end{pmatrix}, a \in R \}\)
\item
  \(V = M_{22}; H = \{A = \begin{pmatrix}a& b \\ -b &c \end{pmatrix},a,b,c\in R\}\)
\item
  \(V = P_4; H = \{ p \in P_4: P(0) = 0\}\)
\item
  \(V = R^3; H = \{ \begin{pmatrix}a\\b\\c\end{pmatrix}, x^2 + y^2 + z^2 - (vt)^2 = 1 \}\)
\end{enumerate}

\hypertarget{combinaciuxf3n-lineal}{%
\subsection{Combinación Lineal}\label{combinaciuxf3n-lineal}}

\begin{enumerate}
\def\labelenumi{\arabic{enumi}.}
\tightlist
\item
  Dados los vectores \(u = (2,1,4)\), \(v = (1,-1,3)\) y
  \(w = (3,2,5)\). Exprese los siguientes vectores como combinaciones
  lineales de \(u\),\(v\) y \(w\).
\end{enumerate}

\begin{enumerate}
\def\labelenumi{\alph{enumi})}
\tightlist
\item
  \((5,5,9)\)
\end{enumerate}

\[
  \left[\begin{matrix}2 & 1 & 3 & 5\\1 & -1 & 2 & 5\\4 & 3 & 5 & 9\end{matrix}\right]
\underrightarrow{ R_{3} \leftrightarrow R_{1} }
\left[\begin{matrix}4 & 3 & 5 & 9\\1 & -1 & 2 & 5\\2 & 1 & 3 & 5\end{matrix}\right]
\underrightarrow{ R_2 + (- \frac{1}{4})R_1 } \\
\left[\begin{matrix}4 & 3 & 5 & 9\\0 & - \frac{7}{4} & \frac{3}{4} & \frac{11}{4}\\2 & 1 & 3 & 5\end{matrix}\right]
\underrightarrow{ R_3 + (- \frac{1}{2})R_1 }
\left[\begin{matrix}4 & 3 & 5 & 9\\0 & - \frac{7}{4} & \frac{3}{4} & \frac{11}{4}\\0 & - \frac{1}{2} & \frac{1}{2} & \frac{1}{2}\end{matrix}\right]
\underrightarrow{ R_{2} \leftrightarrow R_{2} } \\
\left[\begin{matrix}4 & 3 & 5 & 9\\0 & - \frac{7}{4} & \frac{3}{4} & \frac{11}{4}\\0 & - \frac{1}{2} & \frac{1}{2} & \frac{1}{2}\end{matrix}\right]
\underrightarrow{ R_3 + (- \frac{2}{7})R_2 }
\left[\begin{matrix}4 & 3 & 5 & 9\\0 & - \frac{7}{4} & \frac{3}{4} & \frac{11}{4}\\0 & 0 & \frac{2}{7} & - \frac{2}{7}\end{matrix}\right]
\underrightarrow{ R_{3} \leftrightarrow R_{3} } \\
\left[\begin{matrix}4 & 3 & 5 & 9\\0 & - \frac{7}{4} & \frac{3}{4} & \frac{11}{4}\\0 & 0 & \frac{2}{7} & - \frac{2}{7}\end{matrix}\right]
\underrightarrow{ \frac{R_3}{\frac{2}{7}} }
\left[\begin{matrix}4 & 3 & 5 & 9\\0 & - \frac{7}{4} & \frac{3}{4} & \frac{11}{4}\\0 & 0 & 1 & -1\end{matrix}\right]
\underrightarrow{ R_2 + (- \frac{3}{4})R_3 } \\
\left[\begin{matrix}4 & 3 & 5 & 9\\0 & - \frac{7}{4} & 0 & \frac{7}{2}\\0 & 0 & 1 & -1\end{matrix}\right]
\underrightarrow{ R_1 + (-5)R_3 }
\left[\begin{matrix}4 & 3 & 0 & 14\\0 & - \frac{7}{4} & 0 & \frac{7}{2}\\0 & 0 & 1 & -1\end{matrix}\right]
\underrightarrow{ \frac{R_2}{- \frac{7}{4}} } \\
\left[\begin{matrix}4 & 3 & 0 & 14\\0 & 1 & 0 & -2\\0 & 0 & 1 & -1\end{matrix}\right]
\underrightarrow{ R_1 + (-3)R_2 }
\left[\begin{matrix}4 & 0 & 0 & 20\\0 & 1 & 0 & -2\\0 & 0 & 1 & -1\end{matrix}\right]
\underrightarrow{ \frac{R_1}{4} }
\left[\begin{matrix}1 & 0 & 0 & 5\\0 & 1 & 0 & -2\\0 & 0 & 1 & -1\end{matrix}\right]
  \]

\begin{enumerate}
\def\labelenumi{\alph{enumi})}
\setcounter{enumi}{1}
\tightlist
\item
  \((2,0,6)\)
\end{enumerate}

\[
  \left[\begin{matrix}2 & 1 & 3 & 2\\1 & -1 & 2 & 0\\4 & 3 & 5 & 6\end{matrix}\right]
\underrightarrow{ R_{3} \leftrightarrow R_{1} }
\left[\begin{matrix}4 & 3 & 5 & 6\\1 & -1 & 2 & 0\\2 & 1 & 3 & 2\end{matrix}\right]
\underrightarrow{ R_2 + (- \frac{1}{4})R_1 }
\left[\begin{matrix}4 & 3 & 5 & 6\\0 & - \frac{7}{4} & \frac{3}{4} & - \frac{3}{2}\\2 & 1 & 3 & 2\end{matrix}\right]
\underrightarrow{ R_3 + (- \frac{1}{2})R_1 }
\\
\left[\begin{matrix}4 & 3 & 5 & 6\\0 & - \frac{7}{4} & \frac{3}{4} & - \frac{3}{2}\\0 & - \frac{1}{2} & \frac{1}{2} & -1\end{matrix}\right]
\underrightarrow{ R_{2} \leftrightarrow R_{2} }
\left[\begin{matrix}4 & 3 & 5 & 6\\0 & - \frac{7}{4} & \frac{3}{4} & - \frac{3}{2}\\0 & - \frac{1}{2} & \frac{1}{2} & -1\end{matrix}\right]
\underrightarrow{ R_3 + (- \frac{2}{7})R_2 }
\\
\left[\begin{matrix}4 & 3 & 5 & 6\\0 & - \frac{7}{4} & \frac{3}{4} & - \frac{3}{2}\\0 & 0 & \frac{2}{7} & - \frac{4}{7}\end{matrix}\right]
\underrightarrow{ R_{3} \leftrightarrow R_{3} }
\left[\begin{matrix}4 & 3 & 5 & 6\\0 & - \frac{7}{4} & \frac{3}{4} & - \frac{3}{2}\\0 & 0 & \frac{2}{7} & - \frac{4}{7}\end{matrix}\right]
\underrightarrow{ \frac{R_3}{\frac{2}{7}} }
\left[\begin{matrix}4 & 3 & 5 & 6\\0 & - \frac{7}{4} & \frac{3}{4} & - \frac{3}{2}\\0 & 0 & 1 & -2\end{matrix}\right]
\underrightarrow{ R_2 + (- \frac{3}{4})R_3 }
\left[\begin{matrix}4 & 3 & 5 & 6\\0 & - \frac{7}{4} & 0 & 0\\0 & 0 & 1 & -2\end{matrix}\right]
\underrightarrow{ R_1 + (-5)R_3 }
\\
\left[\begin{matrix}4 & 3 & 0 & 16\\0 & - \frac{7}{4} & 0 & 0\\0 & 0 & 1 & -2\end{matrix}\right]
\underrightarrow{ \frac{R_2}{- \frac{7}{4}} }
\left[\begin{matrix}4 & 3 & 0 & 16\\0 & 1 & 0 & 0\\0 & 0 & 1 & -2\end{matrix}\right]
\underrightarrow{ R_1 + (-3)R_2 }
\\
\left[\begin{matrix}4 & 0 & 0 & 16\\0 & 1 & 0 & 0\\0 & 0 & 1 & -2\end{matrix}\right]
\underrightarrow{ \frac{R_1}{4} }
\left[\begin{matrix}1 & 0 & 0 & 4\\0 & 1 & 0 & 0\\0 & 0 & 1 & -2\end{matrix}\right]
  \]

\begin{enumerate}
\def\labelenumi{\alph{enumi})}
\setcounter{enumi}{2}
\tightlist
\item
  \((2,2,3)\)
\end{enumerate}

\[
  \left[\begin{matrix}2 & 1 & 3 & 2\\1 & -1 & 2 & 2\\4 & 3 & 5 & 3\end{matrix}\right]
\underrightarrow{ R_{3} \leftrightarrow R_{1} }
\left[\begin{matrix}4 & 3 & 5 & 3\\1 & -1 & 2 & 2\\2 & 1 & 3 & 2\end{matrix}\right]
\underrightarrow{ R_2 + (- \frac{1}{4})R_1 }
\left[\begin{matrix}4 & 3 & 5 & 3\\0 & - \frac{7}{4} & \frac{3}{4} & \frac{5}{4}\\2 & 1 & 3 & 2\end{matrix}\right]
\underrightarrow{ R_3 + (- \frac{1}{2})R_1 }
\\
\left[\begin{matrix}4 & 3 & 5 & 3\\0 & - \frac{7}{4} & \frac{3}{4} & \frac{5}{4}\\0 & - \frac{1}{2} & \frac{1}{2} & \frac{1}{2}\end{matrix}\right]
\underrightarrow{ R_{2} \leftrightarrow R_{2} }
\left[\begin{matrix}4 & 3 & 5 & 3\\0 & - \frac{7}{4} & \frac{3}{4} & \frac{5}{4}\\0 & - \frac{1}{2} & \frac{1}{2} & \frac{1}{2}\end{matrix}\right]
\underrightarrow{ R_3 + (- \frac{2}{7})R_2 }
\\
\left[\begin{matrix}4 & 3 & 5 & 3\\0 & - \frac{7}{4} & \frac{3}{4} & \frac{5}{4}\\0 & 0 & \frac{2}{7} & \frac{1}{7}\end{matrix}\right]
\underrightarrow{ R_{3} \leftrightarrow R_{3} }
\left[\begin{matrix}4 & 3 & 5 & 3\\0 & - \frac{7}{4} & \frac{3}{4} & \frac{5}{4}\\0 & 0 & \frac{2}{7} & \frac{1}{7}\end{matrix}\right]
\underrightarrow{ \frac{R_3}{\frac{2}{7}} }
\left[\begin{matrix}4 & 3 & 5 & 3\\0 & - \frac{7}{4} & \frac{3}{4} & \frac{5}{4}\\0 & 0 & 1 & \frac{1}{2}\end{matrix}\right]
\underrightarrow{ R_2 + (- \frac{3}{4})R_3 }
\left[\begin{matrix}4 & 3 & 5 & 3\\0 & - \frac{7}{4} & 0 & \frac{7}{8}\\0 & 0 & 1 & \frac{1}{2}\end{matrix}\right]
\underrightarrow{ R_1 + (-5)R_3 }
\\
\left[\begin{matrix}4 & 3 & 0 & \frac{1}{2}\\0 & - \frac{7}{4} & 0 & \frac{7}{8}\\0 & 0 & 1 & \frac{1}{2}\end{matrix}\right]
\underrightarrow{ \frac{R_2}{- \frac{7}{4}} }
\left[\begin{matrix}4 & 3 & 0 & \frac{1}{2}\\0 & 1 & 0 & - \frac{1}{2}\\0 & 0 & 1 & \frac{1}{2}\end{matrix}\right]
\underrightarrow{ R_1 + (-3)R_2 }
\\
\left[\begin{matrix}4 & 0 & 0 & 2\\0 & 1 & 0 & - \frac{1}{2}\\0 & 0 & 1 & \frac{1}{2}\end{matrix}\right]
\underrightarrow{ \frac{R_1}{4} }
\left[\begin{matrix}1 & 0 & 0 & \frac{1}{2}\\0 & 1 & 0 & - \frac{1}{2}\\0 & 0 & 1 & \frac{1}{2}\end{matrix}\right]
  \]

\begin{enumerate}
\def\labelenumi{\alph{enumi})}
\setcounter{enumi}{3}
\tightlist
\item
  \((-1,3,\frac{1}{2})\)
\end{enumerate}

\[
  \left[\begin{matrix}2 & 1 & 3 & -1.0\\1 & -1 & 2 & 3.0\\4 & 3 & 5 & 0.5\end{matrix}\right]
\underrightarrow{ R_{3} \leftrightarrow R_{1} }
\left[\begin{matrix}4 & 3 & 5 & 0.5\\1 & -1 & 2 & 3.0\\2 & 1 & 3 & -1.0\end{matrix}\right]
\underrightarrow{ R_2 + (- \frac{1}{4})R_1 }
\left[\begin{matrix}4 & 3 & 5 & 0.5\\0 & - \frac{7}{4} & \frac{3}{4} & 2.875\\2 & 1 & 3 & -1.0\end{matrix}\right]
\underrightarrow{ R_3 + (- \frac{1}{2})R_1 }
\\
\left[\begin{matrix}4 & 3 & 5 & 0.5\\0 & - \frac{7}{4} & \frac{3}{4} & 2.875\\0 & - \frac{1}{2} & \frac{1}{2} & -1.25\end{matrix}\right]
\underrightarrow{ R_{2} \leftrightarrow R_{2} }
\left[\begin{matrix}4 & 3 & 5 & 0.5\\0 & - \frac{7}{4} & \frac{3}{4} & 2.875\\0 & - \frac{1}{2} & \frac{1}{2} & -1.25\end{matrix}\right]
\underrightarrow{ R_3 + (- \frac{2}{7})R_2 }
\\
\left[\begin{matrix}4 & 3 & 5 & 0.5\\0 & - \frac{7}{4} & \frac{3}{4} & 2.875\\0 & 0 & \frac{2}{7} & -2.07142857142857\end{matrix}\right]
\underrightarrow{ R_{3} \leftrightarrow R_{3} }
\left[\begin{matrix}4 & 3 & 5 & 0.5\\0 & - \frac{7}{4} & \frac{3}{4} & 2.875\\0 & 0 & \frac{2}{7} & -2.07142857142857\end{matrix}\right]
\underrightarrow{ \frac{R_3}{\frac{2}{7}} }
\left[\begin{matrix}4 & 3 & 5 & 0.5\\0 & - \frac{7}{4} & \frac{3}{4} & 2.875\\0 & 0 & 1 & -7.25\end{matrix}\right]
\underrightarrow{ R_2 + (- \frac{3}{4})R_3 }
\left[\begin{matrix}4 & 3 & 5 & 0.5\\0 & - \frac{7}{4} & 0 & 8.3125\\0 & 0 & 1 & -7.25\end{matrix}\right]
\underrightarrow{ R_1 + (-5)R_3 }
\\
\left[\begin{matrix}4 & 3 & 0 & 36.75\\0 & - \frac{7}{4} & 0 & 8.3125\\0 & 0 & 1 & -7.25\end{matrix}\right]
\underrightarrow{ \frac{R_2}{- \frac{7}{4}} }
\left[\begin{matrix}4 & 3 & 0 & 36.75\\0 & 1 & 0 & -4.75\\0 & 0 & 1 & -7.25\end{matrix}\right]
\underrightarrow{ R_1 + (-3)R_2 }
\\
\left[\begin{matrix}4 & 0 & 0 & 51.0\\0 & 1 & 0 & -4.75\\0 & 0 & 1 & -7.25\end{matrix}\right]
\underrightarrow{ \frac{R_1}{4} }
\left[\begin{matrix}1 & 0 & 0 & 12.75\\0 & 1 & 0 & -4.75\\0 & 0 & 1 & -7.25\end{matrix}\right]
  \]

\begin{enumerate}
\def\labelenumi{\arabic{enumi}.}
\setcounter{enumi}{1}
\tightlist
\item
  Si \(p_1 = 2 + x + 4x^2\), \(p_2 = 1 - x + 3x^2\) y
  \(p_3 = 3+2x+5x^2\). Exprese los siguientes polinomios como una
  combinación de \(p_1\), \(p_2\) y \(p_3\).
\end{enumerate}

\begin{enumerate}
\def\labelenumi{\alph{enumi})}
\tightlist
\item
  \(9x+5-5x^2\)
\end{enumerate}

\[
  \left[\begin{matrix}4 & 3 & 5 & -5\\1 & -1 & 2 & 9\\2 & 1 & 3 & 5\end{matrix}\right]
\underrightarrow{ R_{1} \leftrightarrow R_{1} }
\left[\begin{matrix}4 & 3 & 5 & -5\\1 & -1 & 2 & 9\\2 & 1 & 3 & 5\end{matrix}\right]
\underrightarrow{ R_2 + (- \frac{1}{4})R_1 }
\left[\begin{matrix}4 & 3 & 5 & -5\\0 & - \frac{7}{4} & \frac{3}{4} & \frac{41}{4}\\2 & 1 & 3 & 5\end{matrix}\right]
\underrightarrow{ R_3 + (- \frac{1}{2})R_1 }
\\
\left[\begin{matrix}4 & 3 & 5 & -5\\0 & - \frac{7}{4} & \frac{3}{4} & \frac{41}{4}\\0 & - \frac{1}{2} & \frac{1}{2} & \frac{15}{2}\end{matrix}\right]
\underrightarrow{ R_{2} \leftrightarrow R_{2} }
\left[\begin{matrix}4 & 3 & 5 & -5\\0 & - \frac{7}{4} & \frac{3}{4} & \frac{41}{4}\\0 & - \frac{1}{2} & \frac{1}{2} & \frac{15}{2}\end{matrix}\right]
\underrightarrow{ R_3 + (- \frac{2}{7})R_2 }
\\
\left[\begin{matrix}4 & 3 & 5 & -5\\0 & - \frac{7}{4} & \frac{3}{4} & \frac{41}{4}\\0 & 0 & \frac{2}{7} & \frac{32}{7}\end{matrix}\right]
\underrightarrow{ R_{3} \leftrightarrow R_{3} }
\left[\begin{matrix}4 & 3 & 5 & -5\\0 & - \frac{7}{4} & \frac{3}{4} & \frac{41}{4}\\0 & 0 & \frac{2}{7} & \frac{32}{7}\end{matrix}\right]
\underrightarrow{ \frac{R_3}{\frac{2}{7}} }
\left[\begin{matrix}4 & 3 & 5 & -5\\0 & - \frac{7}{4} & \frac{3}{4} & \frac{41}{4}\\0 & 0 & 1 & 16\end{matrix}\right]
\underrightarrow{ R_2 + (- \frac{3}{4})R_3 }
\left[\begin{matrix}4 & 3 & 5 & -5\\0 & - \frac{7}{4} & 0 & - \frac{7}{4}\\0 & 0 & 1 & 16\end{matrix}\right]
\underrightarrow{ R_1 + (-5)R_3 }
\\
\left[\begin{matrix}4 & 3 & 0 & -85\\0 & - \frac{7}{4} & 0 & - \frac{7}{4}\\0 & 0 & 1 & 16\end{matrix}\right]
\underrightarrow{ \frac{R_2}{- \frac{7}{4}} }
\left[\begin{matrix}4 & 3 & 0 & -85\\0 & 1 & 0 & 1\\0 & 0 & 1 & 16\end{matrix}\right]
\underrightarrow{ R_1 + (-3)R_2 }
\\
\left[\begin{matrix}4 & 0 & 0 & -88\\0 & 1 & 0 & 1\\0 & 0 & 1 & 16\end{matrix}\right]
\underrightarrow{ \frac{R_1}{4} }
\left[\begin{matrix}1 & 0 & 0 & -22\\0 & 1 & 0 & 1\\0 & 0 & 1 & 16\end{matrix}\right]
  \]

\begin{enumerate}
\def\labelenumi{\alph{enumi})}
\setcounter{enumi}{1}
\tightlist
\item
  \(2+6x^2\)
\end{enumerate}

\[
  \left[\begin{matrix}4 & 3 & 5 & 6\\1 & -1 & 2 & 0\\2 & 1 & 3 & 2\end{matrix}\right]
\underrightarrow{ R_{1} \leftrightarrow R_{1} }
\left[\begin{matrix}4 & 3 & 5 & 6\\1 & -1 & 2 & 0\\2 & 1 & 3 & 2\end{matrix}\right]
\underrightarrow{ R_2 + (- \frac{1}{4})R_1 }
\left[\begin{matrix}4 & 3 & 5 & 6\\0 & - \frac{7}{4} & \frac{3}{4} & - \frac{3}{2}\\2 & 1 & 3 & 2\end{matrix}\right]
\underrightarrow{ R_3 + (- \frac{1}{2})R_1 }
\\
\left[\begin{matrix}4 & 3 & 5 & 6\\0 & - \frac{7}{4} & \frac{3}{4} & - \frac{3}{2}\\0 & - \frac{1}{2} & \frac{1}{2} & -1\end{matrix}\right]
\underrightarrow{ R_{2} \leftrightarrow R_{2} }
\left[\begin{matrix}4 & 3 & 5 & 6\\0 & - \frac{7}{4} & \frac{3}{4} & - \frac{3}{2}\\0 & - \frac{1}{2} & \frac{1}{2} & -1\end{matrix}\right]
\underrightarrow{ R_3 + (- \frac{2}{7})R_2 }
\\
\left[\begin{matrix}4 & 3 & 5 & 6\\0 & - \frac{7}{4} & \frac{3}{4} & - \frac{3}{2}\\0 & 0 & \frac{2}{7} & - \frac{4}{7}\end{matrix}\right]
\underrightarrow{ R_{3} \leftrightarrow R_{3} }
\left[\begin{matrix}4 & 3 & 5 & 6\\0 & - \frac{7}{4} & \frac{3}{4} & - \frac{3}{2}\\0 & 0 & \frac{2}{7} & - \frac{4}{7}\end{matrix}\right]
\underrightarrow{ \frac{R_3}{\frac{2}{7}} }
\left[\begin{matrix}4 & 3 & 5 & 6\\0 & - \frac{7}{4} & \frac{3}{4} & - \frac{3}{2}\\0 & 0 & 1 & -2\end{matrix}\right]
\underrightarrow{ R_2 + (- \frac{3}{4})R_3 }
\left[\begin{matrix}4 & 3 & 5 & 6\\0 & - \frac{7}{4} & 0 & 0\\0 & 0 & 1 & -2\end{matrix}\right]
\underrightarrow{ R_1 + (-5)R_3 }
\\
\left[\begin{matrix}4 & 3 & 0 & 16\\0 & - \frac{7}{4} & 0 & 0\\0 & 0 & 1 & -2\end{matrix}\right]
\underrightarrow{ \frac{R_2}{- \frac{7}{4}} }
\left[\begin{matrix}4 & 3 & 0 & 16\\0 & 1 & 0 & 0\\0 & 0 & 1 & -2\end{matrix}\right]
\underrightarrow{ R_1 + (-3)R_2 }
\\
\left[\begin{matrix}4 & 0 & 0 & 16\\0 & 1 & 0 & 0\\0 & 0 & 1 & -2\end{matrix}\right]
\underrightarrow{ \frac{R_1}{4} }
\left[\begin{matrix}1 & 0 & 0 & 4\\0 & 1 & 0 & 0\\0 & 0 & 1 & -2\end{matrix}\right]
  \]

\begin{enumerate}
\def\labelenumi{\alph{enumi})}
\setcounter{enumi}{2}
\tightlist
\item
  \(3-2^{-1}x^2 + x\)
\end{enumerate}

\[
  \left[\begin{matrix}4 & 3 & 5 & -0.5\\1 & -1 & 2 & 1.0\\2 & 1 & 3 & 3.0\end{matrix}\right]
\underrightarrow{ R_{1} \leftrightarrow R_{1} }
\left[\begin{matrix}4 & 3 & 5 & -0.5\\1 & -1 & 2 & 1.0\\2 & 1 & 3 & 3.0\end{matrix}\right]
\underrightarrow{ R_2 + (- \frac{1}{4})R_1 }
\left[\begin{matrix}4 & 3 & 5 & -0.5\\0 & - \frac{7}{4} & \frac{3}{4} & 1.125\\2 & 1 & 3 & 3.0\end{matrix}\right]
\underrightarrow{ R_3 + (- \frac{1}{2})R_1 }
\\
\left[\begin{matrix}4 & 3 & 5 & -0.5\\0 & - \frac{7}{4} & \frac{3}{4} & 1.125\\0 & - \frac{1}{2} & \frac{1}{2} & 3.25\end{matrix}\right]
\underrightarrow{ R_{2} \leftrightarrow R_{2} }
\left[\begin{matrix}4 & 3 & 5 & -0.5\\0 & - \frac{7}{4} & \frac{3}{4} & 1.125\\0 & - \frac{1}{2} & \frac{1}{2} & 3.25\end{matrix}\right]
\underrightarrow{ R_3 + (- \frac{2}{7})R_2 }
\\
\left[\begin{matrix}4 & 3 & 5 & -0.5\\0 & - \frac{7}{4} & \frac{3}{4} & 1.125\\0 & 0 & \frac{2}{7} & 2.92857142857143\end{matrix}\right]
\underrightarrow{ R_{3} \leftrightarrow R_{3} }
\left[\begin{matrix}4 & 3 & 5 & -0.5\\0 & - \frac{7}{4} & \frac{3}{4} & 1.125\\0 & 0 & \frac{2}{7} & 2.92857142857143\end{matrix}\right]
\underrightarrow{ \frac{R_3}{\frac{2}{7}} }
\left[\begin{matrix}4 & 3 & 5 & -0.5\\0 & - \frac{7}{4} & \frac{3}{4} & 1.125\\0 & 0 & 1 & 10.25\end{matrix}\right]
\underrightarrow{ R_2 + (- \frac{3}{4})R_3 }
\left[\begin{matrix}4 & 3 & 5 & -0.5\\0 & - \frac{7}{4} & 0 & -6.5625\\0 & 0 & 1 & 10.25\end{matrix}\right]
\underrightarrow{ R_1 + (-5)R_3 }
\\
\left[\begin{matrix}4 & 3 & 0 & -51.75\\0 & - \frac{7}{4} & 0 & -6.5625\\0 & 0 & 1 & 10.25\end{matrix}\right]
\underrightarrow{ \frac{R_2}{- \frac{7}{4}} }
\left[\begin{matrix}4 & 3 & 0 & -51.75\\0 & 1 & 0 & 3.75\\0 & 0 & 1 & 10.25\end{matrix}\right]
\underrightarrow{ R_1 + (-3)R_2 }
\\
\left[\begin{matrix}4 & 0 & 0 & -63.0\\0 & 1 & 0 & 3.75\\0 & 0 & 1 & 10.25\end{matrix}\right]
\underrightarrow{ \frac{R_1}{4} }
\left[\begin{matrix}1 & 0 & 0 & -15.75\\0 & 1 & 0 & 3.75\\0 & 0 & 1 & 10.25\end{matrix}\right]
  \]

\begin{enumerate}
\def\labelenumi{\alph{enumi})}
\setcounter{enumi}{3}
\tightlist
\item
  \(7x-2x^2+1\)
\end{enumerate}

\[
  \left[\begin{matrix}4 & 3 & 5 & -2\\1 & -1 & 2 & 7\\2 & 1 & 3 & 1\end{matrix}\right]
\underrightarrow{ R_{1} \leftrightarrow R_{1} }
\left[\begin{matrix}4 & 3 & 5 & -2\\1 & -1 & 2 & 7\\2 & 1 & 3 & 1\end{matrix}\right]
\underrightarrow{ R_2 + (- \frac{1}{4})R_1 }
\left[\begin{matrix}4 & 3 & 5 & -2\\0 & - \frac{7}{4} & \frac{3}{4} & \frac{15}{2}\\2 & 1 & 3 & 1\end{matrix}\right]
\underrightarrow{ R_3 + (- \frac{1}{2})R_1 }
\\
\left[\begin{matrix}4 & 3 & 5 & -2\\0 & - \frac{7}{4} & \frac{3}{4} & \frac{15}{2}\\0 & - \frac{1}{2} & \frac{1}{2} & 2\end{matrix}\right]
\underrightarrow{ R_{2} \leftrightarrow R_{2} }
\left[\begin{matrix}4 & 3 & 5 & -2\\0 & - \frac{7}{4} & \frac{3}{4} & \frac{15}{2}\\0 & - \frac{1}{2} & \frac{1}{2} & 2\end{matrix}\right]
\underrightarrow{ R_3 + (- \frac{2}{7})R_2 }
\\
\left[\begin{matrix}4 & 3 & 5 & -2\\0 & - \frac{7}{4} & \frac{3}{4} & \frac{15}{2}\\0 & 0 & \frac{2}{7} & - \frac{1}{7}\end{matrix}\right]
\underrightarrow{ R_{3} \leftrightarrow R_{3} }
\left[\begin{matrix}4 & 3 & 5 & -2\\0 & - \frac{7}{4} & \frac{3}{4} & \frac{15}{2}\\0 & 0 & \frac{2}{7} & - \frac{1}{7}\end{matrix}\right]
\underrightarrow{ \frac{R_3}{\frac{2}{7}} }
\left[\begin{matrix}4 & 3 & 5 & -2\\0 & - \frac{7}{4} & \frac{3}{4} & \frac{15}{2}\\0 & 0 & 1 & - \frac{1}{2}\end{matrix}\right]
\underrightarrow{ R_2 + (- \frac{3}{4})R_3 }
\left[\begin{matrix}4 & 3 & 5 & -2\\0 & - \frac{7}{4} & 0 & \frac{63}{8}\\0 & 0 & 1 & - \frac{1}{2}\end{matrix}\right]
\underrightarrow{ R_1 + (-5)R_3 }
\\
\left[\begin{matrix}4 & 3 & 0 & \frac{1}{2}\\0 & - \frac{7}{4} & 0 & \frac{63}{8}\\0 & 0 & 1 & - \frac{1}{2}\end{matrix}\right]
\underrightarrow{ \frac{R_2}{- \frac{7}{4}} }
\left[\begin{matrix}4 & 3 & 0 & \frac{1}{2}\\0 & 1 & 0 & - \frac{9}{2}\\0 & 0 & 1 & - \frac{1}{2}\end{matrix}\right]
\underrightarrow{ R_1 + (-3)R_2 }
\\
\left[\begin{matrix}4 & 0 & 0 & 14\\0 & 1 & 0 & - \frac{9}{2}\\0 & 0 & 1 & - \frac{1}{2}\end{matrix}\right]
\underrightarrow{ \frac{R_1}{4} }
\left[\begin{matrix}1 & 0 & 0 & \frac{7}{2}\\0 & 1 & 0 & - \frac{9}{2}\\0 & 0 & 1 & - \frac{1}{2}\end{matrix}\right]
  \]

\begin{enumerate}
\def\labelenumi{\arabic{enumi}.}
\setcounter{enumi}{2}
\tightlist
\item
  Escriba a \(B\) como una combinación lineal del conjunto de vectores
  A.
\end{enumerate}

\begin{enumerate}
\def\labelenumi{\alph{enumi})}
\tightlist
\item
  \(B = \begin{pmatrix} -1 \\ -2 \\ 4\end{pmatrix}, A = \{\begin{pmatrix}-2 \\ -1 \\ -5\end{pmatrix}, \begin{pmatrix}4 \\ -1 \\ -2\end{pmatrix}, \begin{pmatrix}3 \\ 1 \\ -3\end{pmatrix}\}\)
\end{enumerate}

\[
  \left[\begin{matrix}-2 & 4 & 3 & -1\\-1 & -1 & 1 & -2\\-5 & -2 & -3 & 4\end{matrix}\right]
\underrightarrow{ R_{3} \leftrightarrow R_{1} }
\left[\begin{matrix}-5 & -2 & -3 & 4\\-1 & -1 & 1 & -2\\-2 & 4 & 3 & -1\end{matrix}\right]
\underrightarrow{ R_2 + (- \frac{1}{5})R_1 }
\left[\begin{matrix}-5 & -2 & -3 & 4\\0 & - \frac{3}{5} & \frac{8}{5} & - \frac{14}{5}\\-2 & 4 & 3 & -1\end{matrix}\right]
\underrightarrow{ R_3 + (- \frac{2}{5})R_1 }
\\
\left[\begin{matrix}-5 & -2 & -3 & 4\\0 & - \frac{3}{5} & \frac{8}{5} & - \frac{14}{5}\\0 & \frac{24}{5} & \frac{21}{5} & - \frac{13}{5}\end{matrix}\right]
\underrightarrow{ R_{3} \leftrightarrow R_{2} }
\left[\begin{matrix}-5 & -2 & -3 & 4\\0 & \frac{24}{5} & \frac{21}{5} & - \frac{13}{5}\\0 & - \frac{3}{5} & \frac{8}{5} & - \frac{14}{5}\end{matrix}\right]
\underrightarrow{ R_3 + (\frac{1}{8})R_2 }
\\
\left[\begin{matrix}-5 & -2 & -3 & 4\\0 & \frac{24}{5} & \frac{21}{5} & - \frac{13}{5}\\0 & 0 & \frac{17}{8} & - \frac{25}{8}\end{matrix}\right]
\underrightarrow{ R_{3} \leftrightarrow R_{3} }
\left[\begin{matrix}-5 & -2 & -3 & 4\\0 & \frac{24}{5} & \frac{21}{5} & - \frac{13}{5}\\0 & 0 & \frac{17}{8} & - \frac{25}{8}\end{matrix}\right]
\underrightarrow{ \frac{R_3}{\frac{17}{8}} }
\left[\begin{matrix}-5 & -2 & -3 & 4\\0 & \frac{24}{5} & \frac{21}{5} & - \frac{13}{5}\\0 & 0 & 1 & - \frac{25}{17}\end{matrix}\right]
\underrightarrow{ R_2 + (- \frac{21}{5})R_3 }
\left[\begin{matrix}-5 & -2 & -3 & 4\\0 & \frac{24}{5} & 0 & \frac{304}{85}\\0 & 0 & 1 & - \frac{25}{17}\end{matrix}\right]
\underrightarrow{ R_1 + (3)R_3 }
\\
\left[\begin{matrix}-5 & -2 & 0 & - \frac{7}{17}\\0 & \frac{24}{5} & 0 & \frac{304}{85}\\0 & 0 & 1 & - \frac{25}{17}\end{matrix}\right]
\underrightarrow{ \frac{R_2}{\frac{24}{5}} }
\left[\begin{matrix}-5 & -2 & 0 & - \frac{7}{17}\\0 & 1 & 0 & \frac{38}{51}\\0 & 0 & 1 & - \frac{25}{17}\end{matrix}\right]
\underrightarrow{ R_1 + (2)R_2 }
\\
\left[\begin{matrix}-5 & 0 & 0 & \frac{55}{51}\\0 & 1 & 0 & \frac{38}{51}\\0 & 0 & 1 & - \frac{25}{17}\end{matrix}\right]
\underrightarrow{ \frac{R_1}{-5} }
\left[\begin{matrix}1 & 0 & 0 & - \frac{11}{51}\\0 & 1 & 0 & \frac{38}{51}\\0 & 0 & 1 & - \frac{25}{17}\end{matrix}\right]
  \]

\begin{enumerate}
\def\labelenumi{\alph{enumi})}
\setcounter{enumi}{1}
\tightlist
\item
  \(B = -x^2+2x, A = \{x^2-1, x^2+1, x^2-x-1, x^2+5x\}\): No se puede
  \(\because\) \(r > m\)
\end{enumerate}

\hypertarget{vectores-linealmente-independientes-y-dependientes}{%
\subsection{Vectores Linealmente Independientes y
Dependientes}\label{vectores-linealmente-independientes-y-dependientes}}

\begin{enumerate}
\def\labelenumi{\arabic{enumi}.}
\item
  Determine los valores de \(k\) para que el conjunto
  \(H = \{\begin{pmatrix}k \\ -2 \\3 \end{pmatrix}, \begin{pmatrix}2 \\ -2k \\ -1\end{pmatrix}, \begin{pmatrix}k \\ 0 \\3\end{pmatrix}\}\)
  sea lineal-mente independiente.
\item
  Sea \(V\) el espacio vectorial de todas las funciones con valore real
  definidas sobre la recta real completa. ¿Cuáles de los siguientes
  conjuntos de vectores en \(V\) son lineal-mente dependientes?
\end{enumerate}

\begin{enumerate}
\def\labelenumi{\alph{enumi})}
\item
  \(\{2,4\sin^2{x},\cos^2{x}\}\)
\item
  \(\{x, \cos{x}\}\)
\item
  \(\{1,\sin{x},\sin{2x}\}\)
\item
  \(\{\cos{2x}, \sin^2{x}, \cos^2{x}\}\)
\end{enumerate}

\begin{enumerate}
\def\labelenumi{\arabic{enumi}.}
\setcounter{enumi}{2}
\tightlist
\item
  Para que los valores de \(k\), las siguientes matrices son linealmente
  independientes de \(M_  {22}\)
\end{enumerate}

\begin{enumerate}
\def\labelenumi{\alph{enumi})}
\item
  \(\begin{bmatrix}1 & 0\\ 1 & k\end{bmatrix}\)
\item
  \(\begin{bmatrix}-1 & 0 \\ k & 1\end{bmatrix}\)
\item
  \(\begin{bmatrix}2 & 0 \\ 1 & 3\end{bmatrix}\)
\end{enumerate}

\begin{enumerate}
\def\labelenumi{\arabic{enumi}.}
\setcounter{enumi}{3}
\item
  Construya un conjunto de vectores \(H = \{v_1, v_2, v_3\} \in R^3\)
  tal que sean linealmente independientes y \(v_1^TV_2 = v_2^Tv_3 = 0\).
\item
  Sea \(H = \{v_1, v_2, v_3\} \in R^3\). Demuestre que si
  \(det(H) = det( [v_1, v_2, v_3] ) = 0\), entonces \(H\) es
  lineal-mente independiente.
\end{enumerate}

\hypertarget{bases-y-cambios-de-base}{%
\subsection{Bases y Cambios de Base}\label{bases-y-cambios-de-base}}

\begin{enumerate}
\def\labelenumi{\arabic{enumi}.}
\item
  Determine una base para el espacio de funciones que satisface:
  \(\frac{dy}{dx}-2y=0\).
\item
  Considera las bases \(B = \{u_1,u_2\}\) y \(B' = \{u'_1,u'_2\}\) para
  R\^{}2, donde:
  \(u_1 = \begin{pmatrix}2\\2\end{pmatrix}, u_2 = \begin{pmatrix}4\\-1\end{pmatrix}, u'_1 = \begin{pmatrix}1\\3\end{pmatrix}, u'_2 = \begin{pmatrix}-1\\-1\end{pmatrix}\)
\end{enumerate}

\begin{enumerate}
\def\labelenumi{\alph{enumi})}
\tightlist
\item
  Calcula la matriz de transición de \(B'\) hacia \(B\).
\end{enumerate}

\[
  \left[\begin{matrix}2 & 4 & 1 & -1\\2 & -1 & 3 & -1\end{matrix}\right]
\underrightarrow{ R_{1} \leftrightarrow R_{1} }
\left[\begin{matrix}2 & 4 & 1 & -1\\2 & -1 & 3 & -1\end{matrix}\right]
\underrightarrow{ R_2 + (-1)R_1 }
\left[\begin{matrix}2 & 4 & 1 & -1\\0 & -5 & 2 & 0\end{matrix}\right]
\underrightarrow{ R_{2} \leftrightarrow R_{2} }
\left[\begin{matrix}2 & 4 & 1 & -1\\0 & -5 & 2 & 0\end{matrix}\right]
\underrightarrow{ \frac{R_2}{-5} }
\left[\begin{matrix}2 & 4 & 1 & -1\\0 & 1 & - \frac{2}{5} & 0\end{matrix}\right]
\underrightarrow{ R_1 + (-4)R_2 }
\\
\left[\begin{matrix}2 & 0 & \frac{13}{5} & -1\\0 & 1 & - \frac{2}{5} & 0\end{matrix}\right]
\underrightarrow{ \frac{R_1}{2} }
\left[\begin{matrix}1 & 0 & \frac{13}{10} & - \frac{1}{2}\\0 & 1 & - \frac{2}{5} & 0\end{matrix}\right]
  \]

\begin{enumerate}
\def\labelenumi{\alph{enumi})}
\setcounter{enumi}{1}
\tightlist
\item
  Calcula la matriz de transición de \(B\) hacia \(B'\).
\end{enumerate}

\[
  \left[\begin{matrix}1 & -1 & 2 & 4\\3 & -1 & 2 & -1\end{matrix}\right]
\underrightarrow{ R_{2} \leftrightarrow R_{1} }
\left[\begin{matrix}3 & -1 & 2 & -1\\1 & -1 & 2 & 4\end{matrix}\right]
\underrightarrow{ R_2 + (- \frac{1}{3})R_1 }
\left[\begin{matrix}3 & -1 & 2 & -1\\0 & - \frac{2}{3} & \frac{4}{3} & \frac{13}{3}\end{matrix}\right]
\underrightarrow{ R_{2} \leftrightarrow R_{2} }
\left[\begin{matrix}3 & -1 & 2 & -1\\0 & - \frac{2}{3} & \frac{4}{3} & \frac{13}{3}\end{matrix}\right]
\underrightarrow{ \frac{R_2}{- \frac{2}{3}} }
\left[\begin{matrix}3 & -1 & 2 & -1\\0 & 1 & -2 & - \frac{13}{2}\end{matrix}\right]
\underrightarrow{ R_1 + (1)R_2 }
\\
\left[\begin{matrix}3 & 0 & 0 & - \frac{15}{2}\\0 & 1 & -2 & - \frac{13}{2}\end{matrix}\right]
\underrightarrow{ \frac{R_1}{3} }
\left[\begin{matrix}1 & 0 & 0 & - \frac{5}{2}\\0 & 1 & -2 & - \frac{13}{2}\end{matrix}\right]
\] c) Dado el vector \(w = \begin{pmatrix}3\\-5\end{pmatrix}\), calcula
\([w]_B\) y \([w]_{B'}\).

\[
  [W]_B = 
  \left[\begin{matrix}2 & 4 & 3\\2 & -1 & -5\end{matrix}\right]
\underrightarrow{ R_{1} \leftrightarrow R_{1} }
\left[\begin{matrix}2 & 4 & 3\\2 & -1 & -5\end{matrix}\right]
\underrightarrow{ R_2 + (-1)R_1 }
\left[\begin{matrix}2 & 4 & 3\\0 & -5 & -8\end{matrix}\right]
\underrightarrow{ R_{2} \leftrightarrow R_{2} }
\left[\begin{matrix}2 & 4 & 3\\0 & -5 & -8\end{matrix}\right]
\underrightarrow{ \frac{R_2}{-5} }
\left[\begin{matrix}2 & 4 & 3\\0 & 1 & \frac{8}{5}\end{matrix}\right]
\underrightarrow{ R_1 + (-4)R_2 }
\\
\left[\begin{matrix}2 & 0 & - \frac{17}{5}\\0 & 1 & \frac{8}{5}\end{matrix}\right]
\underrightarrow{ \frac{R_1}{2} }
\left[\begin{matrix}1 & 0 & - \frac{17}{10}\\0 & 1 & \frac{8}{5}\end{matrix}\right]
  \]

\[
  [W]_{B'} = \left[\begin{matrix}1 & -1 & 3\\3 & -1 & -5\end{matrix}\right]
\underrightarrow{ R_{2} \leftrightarrow R_{1} }
\left[\begin{matrix}3 & -1 & -5\\1 & -1 & 3\end{matrix}\right]
\underrightarrow{ R_2 + (- \frac{1}{3})R_1 }
\left[\begin{matrix}3 & -1 & -5\\0 & - \frac{2}{3} & \frac{14}{3}\end{matrix}\right]
\underrightarrow{ R_{2} \leftrightarrow R_{2} }
\left[\begin{matrix}3 & -1 & -5\\0 & - \frac{2}{3} & \frac{14}{3}\end{matrix}\right]
\underrightarrow{ \frac{R_2}{- \frac{2}{3}} }
\left[\begin{matrix}3 & -1 & -5\\0 & 1 & -7\end{matrix}\right]
\underrightarrow{ R_1 + (1)R_2 }
\\
\left[\begin{matrix}3 & 0 & -12\\0 & 1 & -7\end{matrix}\right]
\underrightarrow{ \frac{R_1}{3} }
\left[\begin{matrix}1 & 0 & -4\\0 & 1 & -7\end{matrix}\right]
  \]

\begin{enumerate}
\def\labelenumi{\arabic{enumi}.}
\setcounter{enumi}{2}
\tightlist
\item
  Sean los polinomios \(p_1 = x^2 + x - 2\), \(p_2 = 3x^2-x\), realiza
  los siguientes ejercicios:
\end{enumerate}

\begin{enumerate}
\def\labelenumi{\alph{enumi})}
\tightlist
\item
  \(p_3 = 2p_1 - p_2\)
\end{enumerate}

\[
  2x^2 + 2x - 4 - 3x^2 - x = -x^2+x-4
  \] b) El conjunto \(\{p_1,p_2,p_3\}\), ¿forman una base? Justifica.

\begin{enumerate}
\def\labelenumi{\arabic{enumi}.}
\setcounter{enumi}{3}
\item
  Dado el siguiente sistema de ecuaciones lineales homogéneo:
  \[\begin{align*}-x+3y+z = 0 \\ 2x+2y-z=0\\3x-y-2z=0\end{align*}\]
  Determina la base(si es que existe) del conjunto solución del
  problema.
\item
  Dados los vectores y bases siguientes:
  \[V = \begin{pmatrix}-1\\3\end{pmatrix}, Z = \left\{ \begin{bmatrix}-1\\-3\end{bmatrix}, \begin{bmatrix}1\\0\end{bmatrix} \right\}, W = \left\{ \begin{bmatrix}1\\-1\end{bmatrix}, \begin{bmatrix}0\\1\end{bmatrix} \right\}\]
\end{enumerate}

\begin{enumerate}
\def\labelenumi{\alph{enumi})}
\tightlist
\item
  Calcula \([V]_W\).
\end{enumerate}

\[
  \left[\begin{matrix}1 & 0 & -1\\-1 & 1 & 3\end{matrix}\right]
\underrightarrow{ R_{1} \leftrightarrow R_{1} }
\left[\begin{matrix}1 & 0 & -1\\-1 & 1 & 3\end{matrix}\right]
\underrightarrow{ R_2 + (1)R_1 }
\left[\begin{matrix}1 & 0 & -1\\0 & 1 & 2\end{matrix}\right]
\underrightarrow{ R_{2} \leftrightarrow R_{2} }
\left[\begin{matrix}1 & 0 & -1\\0 & 1 & 2\end{matrix}\right]
\underrightarrow{ \frac{R_2}{1} }
\left[\begin{matrix}1 & 0 & -1\\0 & 1 & 2\end{matrix}\right]
\underrightarrow{ R_1 + (0)R_2 }
\\
\left[\begin{matrix}1 & 0 & -1\\0 & 1 & 2\end{matrix}\right]
\underrightarrow{ \frac{R_1}{1} }
\left[\begin{matrix}1 & 0 & -1\\0 & 1 & 2\end{matrix}\right]
\]

\begin{enumerate}
\def\labelenumi{\alph{enumi})}
\setcounter{enumi}{1}
\tightlist
\item
  Calcula la matriz de transición de \(W\) hacia \(Z\).
\end{enumerate}

\[
  \left[\begin{matrix}-1 & 1 & 1 & 0\\-3 & 0 & -1 & 1\end{matrix}\right]
\underrightarrow{ R_{2} \leftrightarrow R_{1} }
\left[\begin{matrix}-3 & 0 & -1 & 1\\-1 & 1 & 1 & 0\end{matrix}\right]
\underrightarrow{ R_2 + (- \frac{1}{3})R_1 }
\left[\begin{matrix}-3 & 0 & -1 & 1\\0 & 1 & \frac{4}{3} & - \frac{1}{3}\end{matrix}\right]
\underrightarrow{ R_{2} \leftrightarrow R_{2} }
\left[\begin{matrix}-3 & 0 & -1 & 1\\0 & 1 & \frac{4}{3} & - \frac{1}{3}\end{matrix}\right]
\underrightarrow{ \frac{R_2}{1} }
\left[\begin{matrix}-3 & 0 & -1 & 1\\0 & 1 & \frac{4}{3} & - \frac{1}{3}\end{matrix}\right]
\underrightarrow{ R_1 + (0)R_2 }
\\
\left[\begin{matrix}-3 & 0 & -1 & 1\\0 & 1 & \frac{4}{3} & - \frac{1}{3}\end{matrix}\right]
\underrightarrow{ \frac{R_1}{-3} }
\left[\begin{matrix}1 & 0 & \frac{1}{3} & - \frac{1}{3}\\0 & 1 & \frac{4}{3} & - \frac{1}{3}\end{matrix}\right]
\]

\begin{enumerate}
\def\labelenumi{\alph{enumi})}
\setcounter{enumi}{2}
\tightlist
\item
  Calcula \([V]_Z\) utilizando la matriz de transición del inciso
  anterior.
\end{enumerate}

\[
  \left[\begin{matrix}\frac{1}{3} & - \frac{1}{3}\\\frac{4}{3} & - \frac{1}{3}\end{matrix}\right] \left[\begin{matrix}-1 \\ 3\end{matrix}\right] = \left[\begin{matrix}-\frac{2}{3} \\ \frac{1}{3}\end{matrix}\right]
  \]

\begin{enumerate}
\def\labelenumi{\arabic{enumi}.}
\setcounter{enumi}{5}
\tightlist
\item
  Dado el siguiente conjunto de vectores:
  \((\frac{1}{2},-7,0), (-\frac{1}{3},0,2),(0,-7,\frac{1}{5})\).
\end{enumerate}

\begin{enumerate}
\def\labelenumi{\alph{enumi})}
\tightlist
\item
  Determine si el conjunto genera a \(R^3\).
\item
  Genere un espacio vectorial de 3 elementos usando el conjunto de
  vectores.
\item
  Con los vectores, construya un sistema de ecuaciones lineales
  homogéneo y determine la base de las soluciones del sistema.
\end{enumerate}

\begin{enumerate}
\def\labelenumi{\arabic{enumi}.}
\setcounter{enumi}{6}
\item
  Sea \(V = \begin{pmatrix}-1\\5\end{pmatrix}\) y
  \([V]_W = \begin{pmatrix}-2\\7\end{pmatrix}\) determine la base \(W\),
  sabiendo que
  \(W = \left\{ \begin{pmatrix}x\\2y\end{pmatrix}, \begin{pmatrix}-y\\-3x\end{pmatrix} \right\}\)
\item
  Sea \([V]_S = \begin{pmatrix}-\frac{1}{2}\\2\end{pmatrix}\) y
  \([V]_W = \begin{pmatrix}-3\\7\end{pmatrix}\), sabiendo que
  \(W = \left\{ \begin{pmatrix}3\\y\end{pmatrix}, \begin{pmatrix}x\\5\end{pmatrix} \right\}\)
  y
  \(S = \left\{ \begin{pmatrix}1\\0\end{pmatrix}, \begin{pmatrix}0\\1\end{pmatrix} \right\}\),
  determine:
\end{enumerate}

\begin{enumerate}
\def\labelenumi{\alph{enumi})}
\tightlist
\item
  \(W\)
\item
  Matriz de transición de la base \(S\) hacia \(W\)
\item
  Matriz de transición de la base \(W\) hacia \(S\).
\end{enumerate}

\begin{enumerate}
\def\labelenumi{\arabic{enumi}.}
\setcounter{enumi}{8}
\tightlist
\item
  Calcular las coordenadas de
  \(V = \begin{pmatrix} -2 \\ 3 \\ 3 \end{pmatrix}\) en términos de base
  \(H = \left\{ \begin{pmatrix} 1 \\ -2 \\3 \end{pmatrix}, \begin{pmatrix}2 \\ -2 \\ -1 \end{pmatrix}, \begin{pmatrix}1\\1\\4\end{pmatrix} \right\}\).
  Calcule \(||V||\) y \(||V_H||\). ¿Porué \(||V|| \ne ||V_H||\)?
\end{enumerate}

\end{document}
